\documentclass[12pt]{article}
\usepackage{graphicx} % For including images
\usepackage{fancyhdr} % For custom headers and footers
\usepackage{amsmath} % For math equations
\usepackage{listings} % For code listings
\usepackage{xcolor} % For coloring code
\usepackage{geometry} % For adjusting page margins
\geometry{a4paper, margin=1in}

% Define the header and footer
\pagestyle{fancy}
\fancyhf{}
\fancyhead[L]{\textbf{Drici Mohammed Hachem - Group A5}}
\fancyfoot[C]{\thepage}

% Define code listing style for C
\lstset{
	language=C, % Set language to C
	basicstyle=\small\ttfamily, % Basic font style
	numbers=left, % Add line numbers on the left
	numberstyle=\tiny\color{gray}, % Style for line numbers
	backgroundcolor=\color{white}, % Background color
	keywordstyle=\color{blue}, % Color for keywords (e.g., int, if, return)
	commentstyle=\color{green}, % Color for comments (e.g., // or /* ... */)
	stringstyle=\color{red}, % Color for strings (e.g., "Hello, World!")
	showstringspaces=false, % Don't show spaces in strings
	frame=single, % Add a frame around the code
	breaklines=true, % Automatically break long lines
	tabsize=4, % Set tab size to 4 spaces
	captionpos=b, % Caption position (bottom)
	xleftmargin=0pt, % Remove left margin
	framexleftmargin=0pt, % Align frame with the left margin
	morekeywords={printf, scanf, main, include} % Add additional C-specific keywords
}

% Exercise Template
% \section*{Exercise 1}
% \vspace{0.5cm}
% \textbf{Solution:} Explain your solution here. You can include mathematical equations if needed:
% \[
% E = mc^2
% \]
% \vspace{1cm}
% \textbf{Code:} Write your code below using the \texttt{lstlisting} environment.
% \begin{lstlisting}
%	# Your code here
%	def hello_world():
%	print("Hello, World!")
%	
%	hello_world()
% \end{lstlisting}
% \vspace{2cm}

\begin{document}
	
	% First Page
	\begin{titlepage}
		\centering
		\vspace*{2cm}
		
		\textbf{\Large People's Democratic Republic of Algeria}\\
		\vspace{0.5cm}
		\textbf{\large Echahid Hamma Lakhdar University}\\
		
		\vspace{1cm}
		\includegraphics[width=10cm]{university_logo.png} % Replace with your university logo file name
		
		\vspace{1.5cm}
		\rule{\textwidth}{1pt}
		\vspace{0.5cm}
		
		\textbf{\Large DS \& ALGO : Series 02}\\
		\vspace{0.5cm}
		\rule{\textwidth}{1pt}
		\vspace{0.5cm}
		
		\vspace{1.5cm}
		\textbf{\Huge Drici Mohammed Hachem}\\
		\vspace{0.5cm}
		\textbf{\Large Group A5}\\
		
		\vfill
	\end{titlepage}
	
	% Second Page
	\newpage
	\section*{Exercise 1}
	 \vspace{0.5cm}
	 
	 \textbf{Code:} 
\begin{lstlisting}
#include<stdio.h>

int main() {
	//first version
	int a,b,i,prod = 0;
	do {
		printf("Enter two positive integers a,b : ");
		scanf("%d %d" , &a , &b);
		if(a < 1 || b < 1) {
			printf("Error! both integers must be positive");
		}
	}while(a < 1 || b < 1);
	for(i = 0 ; i < b; i++ ) {
		prod += a;
	}
	printf("%d x %d = %d \n", a , b , prod);
	//second version
	int A,A_copy,B,div = 0;
	do {
		printf("Enter two positive integers a,b such that a > b : ");
		scanf("%d %d" , &A , &B);
		if(A < 1 || B < 1 || B > A) {
			printf("Error! both integers must be positive and a > b");
		}
	}while(A < 1 || B < 1 || B > A);
	A_copy = A;
	while(A_copy >= B) {
		A_copy -= B;
		div++;
	}
	printf("The integer division %d / %d = %d \n ", A , B , div);
	return 0;
}
		 
\end{lstlisting}
\vspace{1cm}
	

\section*{Exercise 2}
\vspace{0.5cm}

\textbf{Code:} 
\begin{lstlisting}
#include<stdio.h>

int main() {
	int n,fact = 1,i;
	do {
		printf("Enter a positive integer : ");
		scanf("%d" , &n);
		if(n < 0) {
			printf("Error! the integer must be positive");
		}
	}while(n < 0);
	if(n == 0 || n == 1) {
		printf("%d ! = 1",n);
	}
	else {
		for(i = 2; i <= n ; i++) {
			fact *= i;
		}
		printf("%d! = %d ",n , fact);
	}
	return 0;
}
	
\end{lstlisting}
\vspace{1cm}

\section*{Exercise 3}
\vspace{0.5cm}

\textbf{Code:} 
\begin{lstlisting}
#include<stdio.h>

int main() {
	int n , S_1=0;
	float a , x , S_2 = 0 , S_3 = 0, S_4 = 0;
	//S_1
	do{
		printf("Enter an odd number n between 20 and 100: ");
		scanf("%d",&n);
		if(n < 20 || n > 100 || n % 2 != 1) {
			printf("The number must me odd and between 20 and 100");
		}
	}while(n < 20 || n > 100 || n % 2 != 1);
	int i = 1;
	int sign = 1;
	while(i <= n) {
		S_1 += i * sign;
		sign = -sign;
		i += 2;
	}
	printf("S_1 = %d \n",S_1);
	//S_2
	do{
		printf("Enter an integer n between 10 and 50");
		scanf("%d",&n);
		if(n < 10 || n > 50 ) {
			printf("The number n must be between 10 and 50");
		}
	}while(n < 10 || n > 50);
	for(i = 1 ; i <= n ; i++) {
		S_2 += (float) i / (2*i-1);
	}
	printf("S_2 = %f \n",S_2);
	//S_3
	do{
		printf("Enter an integer n strictly greater than 5 and a real number x");
		scanf("%d %f",&n,&x);
		if(n < 5) {
			printf("n must be strictly greater than 5");
		}
	}while(n < 5);
	int fact = 1;
	for(i = 1 ; i <= n+1 ; i++) {
		S_3 = (float)(x+i-1) / fact;
		fact *= (i+1);
	}
	printf("S_3 = %f \n",S_3);
	//S_4
	do{
		printf("Enter two real numbers a , x and an integer n greater than 10 ");
		scanf("%f %f %d",&a,&x,&n);
		if(n < 10) {
			printf("The number n must be greater than 10");
		}
	}while(n < 10);
	for(i = 2 ; i < n ; i++) {
		x *= x;
	}
	for( i = 1 ; i <= n ; i++) {
		S_4 += a * x;
		a *= a;
		x /= x;
	}
	printf("S_4 = %f", S_4);
	return 0;
}
\end{lstlisting}
\vspace{1cm}

\section*{Exercise 4}
\vspace{0.5cm}

\textbf{Code:}
\begin{lstlisting}
#include<stdio.h>

int main() {
	int n,m,number,div,div_sum;
	do{
		printf("Enter two strictly positive integers n , m such that n < m : ");
		scanf("%d %d",&n , &m);
		if(n < 0 || m < 0 || n > m) printf("n , m must be strictly postiive and n < m");
	}while(n < 0 || m < 0 || n > m);
	
	for(number  = n ; number <= m ; number++) {
		div_sum = 1;
		for(div = 2 ; div <= number/2 ; div++) {
			if(number % div == 0) {
				div_sum += div;
			}
		}
		if(div_sum == number) printf("%d is a perfect number \n",number);
	}
	return 0;
}
\end{lstlisting}
\vspace{1cm}
\section*{Exercise 5}
\vspace{0.5cm}

\textbf{Code:}
\begin{lstlisting}
#include <stdio.h>

int main() {
	int n;
	float u = 2.0, v, sum = 0.0;
	
	for (n = 0; n < 100; n++) {
		v = 1 - (1 / u);   
		sum += v;        
		u = 1 + (1 / v);    
	}
	printf(": %f \n", sum);
	return 0;
}
\end{lstlisting}
\vspace{1cm}
\section*{Exercise 6}
\vspace{0.5cm}

\textbf{Code:} 
\begin{lstlisting}
#include<stdio.h> 

int main() {
	int n,n_copy,possible_digits,digit,counter;
	do{
		printf("Enter a strictly positive integer n : ");
		scanf("%d",&n);
		if(n < 0) printf("n must be strictly positive");
	}while(n < 0);
	for(possible_digits = 0 ; possible_digits < 10 ; possible_digits++){
		counter = 0;
		n_copy = n;
		while(n_copy >= 1) {
			digit = n_copy % 10;
			n_copy /= 10;
			if(possible_digits == digit) counter++;
		}
		if(counter != 0) printf("The digit %d appeared %d times \n",possible_digits,counter);
	}
	return 0;
}
\end{lstlisting}
\vspace{1cm}



\end{document}