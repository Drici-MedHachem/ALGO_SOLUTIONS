\documentclass[12pt]{article}
\usepackage{graphicx} % For including images
\usepackage{fancyhdr} % For custom headers and footers
\usepackage{amsmath} % For math equations
\usepackage{listings} % For code listings
\usepackage{xcolor} % For coloring code
\usepackage{geometry} % For adjusting page margins
\geometry{a4paper, margin=1in}

% Define the header and footer
\pagestyle{fancy}
\fancyhf{}
\fancyhead[L]{\textbf{Drici Mohammed Hachem - Group A5}}
\fancyfoot[C]{\thepage}

% Define code listing style for C
\lstset{
	language=C, % Set language to C
	basicstyle=\small\ttfamily, % Basic font style
	numbers=left, % Add line numbers on the left
	numberstyle=\tiny\color{gray}, % Style for line numbers
	backgroundcolor=\color{white}, % Background color
	keywordstyle=\color{blue}, % Color for keywords (e.g., int, if, return)
	commentstyle=\color{green}, % Color for comments (e.g., // or /* ... */)
	stringstyle=\color{red}, % Color for strings (e.g., "Hello, World!")
	showstringspaces=false, % Don't show spaces in strings
	frame=single, % Add a frame around the code
	breaklines=true, % Automatically break long lines
	tabsize=4, % Set tab size to 4 spaces
	captionpos=b, % Caption position (bottom)
	xleftmargin=0pt, % Remove left margin
	framexleftmargin=0pt, % Align frame with the left margin
	morekeywords={printf, scanf, main, include} % Add additional C-specific keywords
}

% Exercise Template
% \section*{Exercise 1}
% \vspace{0.5cm}
% \textbf{Solution:} Explain your solution here. You can include mathematical equations if needed:
% \[
% E = mc^2
% \]
% \vspace{1cm}
% \textbf{Code:} Write your code below using the \texttt{lstlisting} environment.
% \begin{lstlisting}
%	# Your code here
%	def hello_world():
%	print("Hello, World!")
%	
%	hello_world()
% \end{lstlisting}
% \vspace{2cm}

\begin{document}
	
	% First Page
	\begin{titlepage}
		\centering
		\vspace*{2cm}
		
		\textbf{\Large People's Democratic Republic of Algeria}\\
		\vspace{0.5cm}
		\textbf{\large Echahid Hamma Lakhdar University}\\
		
		\vspace{1cm}
		\includegraphics[width=10cm]{university_logo.png} % Replace with your university logo file name
		
		\vspace{1.5cm}
		\rule{\textwidth}{1pt}
		\vspace{0.5cm}
		
		\textbf{\Large DS \& ALGO : Series 01}\\
		\vspace{0.5cm}
		\rule{\textwidth}{1pt}
		\vspace{0.5cm}
		
		\vspace{1.5cm}
		\textbf{\Huge Drici Mohammed Hachem}\\
		\vspace{0.5cm}
		\textbf{\Large Group A5}\\
		
		\vfill
	\end{titlepage}
	
	% Second Page
	\newpage
	\section*{Exercise 1}
	 \vspace{0.5cm}
	 
	 \textbf{Code:} Using a third variable $c$
\begin{lstlisting}
#include <stdio.h>

int main() {
	int a,b,c;
	printf("Enter the value of a and b : ");
	scanf("%d %d" ,&a,&b);
	printf("Before swapping : a = %d , b = %d \n",a,b);
	c = a;
	a = b;
 	b = c;
 	printf("After swapping : a = %d , b = %d",a,b);
 	return 0;
 }
		 
\end{lstlisting}
	\vspace{1cm}
	
	 \textbf{Code:} Without using a third variable
\begin{lstlisting}
#include <stdio.h>

int main() {
	int a,b;
	printf("Enter the value of a and b : ");
	scanf("%d %d" ,&a,&b);
	printf("Before swapping : a = %d , b = %d \n",a,b);
	a = a+b;
	b = a-b;
	a = a-b;
	printf("After swapping : a = %d , b = %d",a,b);
	return 0;
}
\end{lstlisting}

\section*{Exercise 2}
\vspace{0.5cm}

\textbf{Code:} 
\begin{lstlisting}
#include <stdio.h>

int main() {
	int a;
	printf("Enter the value of a in Octal : ");
	scanf("%o" ,&a);
	printf("in decimal a = %d, in hexadecimal a = %x",a,a);
	return 0;
}
	
\end{lstlisting}
\vspace{1cm}

\section*{Exercise 3}
\vspace{0.5cm}

\textbf{Code:} 
\begin{lstlisting}
#include<stdio.h>

int main() {
	//Version 1 Integer division case
	int a,b,c;
	printf("Enter the value of a and b : ");
	scanf("%d %d" ,&a,&b);
	c = a/b;
	printf("The integer division a/b = %d",c);
	//Version 2 Real division with precision 3
	int a,b;
	float c;
	printf("Enter the value of a and b : ");
	scanf("%d %d" ,&a,&b);
	c = (float) a/b;
	printf("The integer division a/b = %.3f",c);
	return 0;
}
\end{lstlisting}
\vspace{1cm}

\section*{Exercise 4}
\vspace{0.5cm}

\textbf{Code:}
\begin{lstlisting}
#include <stdio.h>
#include <math.h>
int main(){
	float P,S,R;
	printf("Enter the value of the circle radius R : ");
	scanf("%f" ,&R);
	S = M_PI*R*R;
	P = 2*M_PI*R;
	printf("The Area of the circle: %f\n",S);
	printf("The Perimeter of the circle: %f\n",P);
	return 0;
}
\end{lstlisting}
\vspace{1cm}
\section*{Exercise 5}
\vspace{0.5cm}

\textbf{Code:}
\begin{lstlisting}
#include <stdio.h>
int main() {
	int A, B, C, D;
	A = 4, B = 1, C = A++ > B || B++ != 2;
	printf("Resultat 1 : A = %d B = %d C = %d\n", A, B, C);
	// Resultat 1 : A = 5 B = 1 C = 1
	A = 4 ; B = 1 ; C = ++A == 5 && ++B == 2;
	printf("Resultat 2 : A = %d B = %d C = %d\n", A, B, C);
	// Resultat 2 : A = 5 B = 2 C = 1
	A= 1, D = ++A == (B = C = 2) ;
	printf ("Reaultat 3 : A = %d B = %d C = %d D = %d\n", A, B, C, D);
	// Reaultat 3 : A = 2 B = 2 C = 2 D = 1
	A = B = C = 10 ; A += B += C ;
	printf("Resultat 4 : A = %d B = %d C = %d\n", A, B, C);
	// Resultat 4 : A = 30 B = 20 C = 10
	A = 5; C = A << 1; B = A >> 2;
	printf("Resultat 5 : A = %d B = %d C = %d\n", A, B, C);
	// Resultat 5 : A = 5 B = 1 C = 10
	A= 4, B = 1; C = A & B; A = B | 2; B ^= 4;
	printf("Resultat 6 : A = %d B = %d C = %d\n", A, B, C);
	// Resultat 6 : A = 3 B = 5 C = 0
	char X = 'C' ;
	printf("Entier = %d,\n Octal = %o,\n Hexa = %x\n", X, X, X);
	// Entier = 67, Octal = 103, Hexa = 43
	int T = "ABC";
	printf(" T comme entier = %d \n", T) ;
	// ???
	return 0;
}
\end{lstlisting}
\vspace{1cm}
\section*{Exercise 6}
\vspace{0.5cm}

\textbf{Code:} 
\begin{lstlisting}
#include <stdio.h>
int main() {
	//Version 1 Testing point inside a rectangle
	float A_X ,A_Y ,BL_X ,BL_Y ,TR_X ,TR_Y;
	printf("Enter the coordinates of the bottom left corner of the
	rectangle: ");
	scanf("%f %f" ,&BL_X ,&BL_Y);
	printf("Enter the coordinates of the top right corner of the
	rectangle: ");
	scanf("%f %f" ,&TR_X ,&TR_Y);
	printf("Enter the coordinates of the point: ");
	scanf("%f %f" ,&A_X ,&A_Y);
	if (A_X >= BL_X && A_X <= TR_X && A_Y >= BL_Y && A_Y <= TR_Y) {
		printf("The point is inside the rectangle \n");
		} else {
		printf("The point is outside the rectangle \n");
		}
	//Version 2 Testing point inside a circle
	float A_X ,A_Y ,O_X ,O_Y ,R;
	printf("Enter the coordinates of the point A: ");
	scanf("%f %f" ,&A_X ,&A_Y);
	printf("Enter the coordinates of the point center point O: ");
	scanf("%f %f" ,&O_X ,&O_Y);
	printf("Enter the radius of the circle: ");
	scanf("%f" ,&R);
	if ((A_X - O_X)*(A_X - O_X) + (A_Y - O_Y)*(A_Y - O_Y) <= R*R) {
		printf("The point A is inside the circle\n");
		} else {
		printf("The point A is outside the circle\n");
		}
	return 0;
}
\end{lstlisting}
\vspace{1cm}
\newpage
\section*{Exercise 7}
\vspace{0.5cm}

\textbf{Code:} 
\begin{lstlisting}
#include <stdio.h>
int main() {
	//Version 1 Using Conditional Instructions
	int a,b,c;
	printf("Enter the value of a, b and c: ");
	scanf("%d %d %d", &a, &b, &c);
	if(a >= b) {
		if(a >= c) {
			printf("a = %d is the largest number\n",a);
			} else {
			printf("c = %d is the largest number\n",c);
			}
		} else {
		if(b >= c) {
			printf("b = %d is the largest number\n",b);
			} else {
			printf("c = %d is the largest number\n",c);
			}
		}
	//Version 2 Using Ternary Operator
	int a,b,c,max;
	printf("Enter the value of a, b and c: ");
	scanf("%d %d %d", &a, &b, &c);
	max = (a >= b) ? (a >= c ? a : c) : (b >= c ? b : c);
	printf("The maximum value is %d\n",max);
	return 0;
}
\end{lstlisting}
\vspace{1cm}

\section*{Exercise 8}
\vspace{0.5cm}

\textbf{Code:} 
\begin{lstlisting}
#include <stdio.h>
int main() {
int A,B,C,D,E;
A = 5; B= (A<0) ? 1 ,2 ,3:10 ,20 ,30; printf("B=%d\n",B); /*B = 10*/
A = -5; B = (A<0) ? 1 ,2 ,3:10 ,20 ,30; printf("B=%d\n",B); /*B = 3*/
A = 5; C= (A<0) ? (1,2,3) :(10 ,20 ,30); printf("C=%d\n",C); /*C = 30*/
A = -5; C= (A<0) ? (1,2,3) :(10 ,20 ,30); printf("C=%d\n",C); /*C = 3*/
D = -1,-2,-3; printf("D=%d\n",D); /*D = -1*/
E = (-1,-2,-3); printf("E=%d\n",E); /*E = -3*/
return 0;
}
\end{lstlisting}
\vspace{1cm}
\section*{Exercise 9}
\vspace{0.5cm}

\textbf{Code:} 
\begin{lstlisting}
#include <stdio.h>
int main() {
	//Version 1 Using Conditional Instructions
	int age;
	printf("Enter the kid's age: ");
	scanf("%d", &age);
	if (age > 14 || age < 6) {
		printf("The age must be between 6 and 14 to be in junior categories\n"); 
	} else {
		printf("Poussin");
		else if(age >= 8 && age <= 9) {
		printf("Pupille");
		else if(age >= 10 && age <= 11) {
		printf("Minime");
		else{
		printf("Cadet");
	}
	//Version 2 Using Switch Cases
	int age;
	printf("Enter the kid's age: ");
	scanf("%d", &age);
	if (age > 14 || age < 6) {
		printf("The age must be between 6 and 14 to be in junior categories\n"); 
	} else {
	switch (age) {
		case 6: case 7:
		printf("Poussin");
		break;
		case 8: case 9:
		printf("Pupille");
		break;
		case 10: case 11:
		printf("Minime");
		break;
		default:
		printf("Cadet");
		break;
		}
	}
	return 0;
}
\end{lstlisting}
\vspace{1cm}

\section*{Exercise 10}
\vspace{0.5cm}

\textbf{Code:} 
\begin{lstlisting}
#include <stdio.h>
#include <math.h>
int main() {
	float A, B, C;
	float delt , x_1 , x_2 , re , im;
	printf("Enter coefficients A, B, and C: ");
	scanf("%f %f %f", &A, &B, &C);
	if (A == 0) {
		if (B == 0) {
			printf("No solution\n");
			} else {
			printf("The solution: x = %.2f\n", -C / B);
			}
		} else {
		delt = B * B - 4 * A * C;
		if (delt > 0) {
			x_1 = (-B + sqrt(delt)) / (2 * A);
			x_2 = (-B - sqrt(delt)) / (2 * A);
			printf("The solutions : x1 = %.2f, x2 = %.2f\n", x_1 ,
			x_2);
			} else if (delt == 0) {
			x_1 = -B / (2 * A);
			printf("The solution : x = %.2f\n", x_1);
			} else {
			re = -B / (2 * A);
			im = sqrt(-delt) / (2 * A);
			printf("The complex olutions: x1 = %.2f + %.2fi , x2 =
			%.2f - %.2fi\n",re ,im ,re ,im);
			}
		}
		return 0;
}
\end{lstlisting}
\vspace{1cm}

\section*{Exercise 11}
\vspace{0.5cm}

\textbf{Code:} 
\begin{lstlisting}
#include <stdio.h>
int main() {
	int birth_day , birth_month , birth_year;
	int current_day , current_month , current_year;
	int age;
	printf("Enter your birth date : ");
	scanf("%d %d %d", &birth_day , &birth_month , &birth_year);
	printf("Enter today 's date : ");
	scanf("%d %d %d", &current_day , &current_month , &current_year);
	age = current_year - birth_year;
	if (current_month < birth_month || (current_month == birth_month
	&& current_day < birth_day)) {
		age --;
		}
	printf("Your age is: %d years , Happy Birth Day!", age);
	return 0;
}
\end{lstlisting}
\vspace{1cm}


\end{document}