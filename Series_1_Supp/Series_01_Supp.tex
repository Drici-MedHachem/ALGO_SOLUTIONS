\documentclass[12pt]{article}
\usepackage{graphicx} % For including images
\usepackage{fancyhdr} % For custom headers and footers
\usepackage{amsmath} % For math equations
\usepackage{listings} % For code listings
\usepackage{xcolor} % For coloring code
\usepackage{geometry} % For adjusting page margins
\geometry{a4paper, margin=1in}

% Define the header and footer
\pagestyle{fancy}
\fancyhf{}
\fancyhead[L]{\textbf{Drici Mohammed Hachem - Group A5}}
\fancyfoot[C]{\thepage}

% Define code listing style for C
\lstset{
	language=C, % Set language to C
	basicstyle=\small\ttfamily, % Basic font style
	numbers=left, % Add line numbers on the left
	numberstyle=\tiny\color{gray}, % Style for line numbers
	backgroundcolor=\color{white}, % Background color
	keywordstyle=\color{blue}, % Color for keywords (e.g., int, if, return)
	commentstyle=\color{green}, % Color for comments (e.g., // or /* ... */)
	stringstyle=\color{red}, % Color for strings (e.g., "Hello, World!")
	showstringspaces=false, % Don't show spaces in strings
	frame=single, % Add a frame around the code
	breaklines=true, % Automatically break long lines
	tabsize=4, % Set tab size to 4 spaces
	captionpos=b, % Caption position (bottom)
	xleftmargin=0pt, % Remove left margin
	framexleftmargin=0pt, % Align frame with the left margin
	morekeywords={printf, scanf, main, include} % Add additional C-specific keywords
}

% Exercise Template
% \section*{Exercise 1}
% \vspace{0.5cm}
% \textbf{Solution:} Explain your solution here. You can include mathematical equations if needed:
% \[
% E = mc^2
% \]
% \vspace{1cm}
% \textbf{Code:} Write your code below using the \texttt{lstlisting} environment.
% \begin{lstlisting}
%	# Your code here
%	def hello_world():
%	print("Hello, World!")
%	
%	hello_world()
% \end{lstlisting}
% \vspace{2cm}

\begin{document}
	
	% First Page
	\begin{titlepage}
		\centering
		\vspace*{2cm}
		
		\textbf{\Large People's Democratic Republic of Algeria}\\
		\vspace{0.5cm}
		\textbf{\large Echahid Hamma Lakhdar University}\\
		
		\vspace{1cm}
		\includegraphics[width=10cm]{university_logo.png} % Replace with your university logo file name
		
		\vspace{1.5cm}
		\rule{\textwidth}{1pt}
		\vspace{0.5cm}
		
		\textbf{\Large DS \& ALGO : CONDITIONAL INSTRUCTIONS SERIES}\\
		\vspace{0.5cm}
		\rule{\textwidth}{1pt}
		\vspace{0.5cm}
		
		\vspace{1.5cm}
		\textbf{\Huge Drici Mohammed Hachem}\\
		\vspace{0.5cm}
		\textbf{\Large Group A5}\\
		
		\vfill
	\end{titlepage}
	
	% Second Page
	\newpage
	\section*{Exercise 1}
	 \vspace{0.5cm}
	 
	 \textbf{Code:}
\begin{lstlisting}
#include<stdio.h>

int main() {
	int hours;
	int cost;
	scanf("%d", &hours);
	if(hours <= 0) {
		printf("Error\n");
	} else {
		if(hours <= 5) {
			cost = hours * 50;
		} else if (hours <= 10) {
			cost = 5 * 50 + (hours - 5) * 40;
		} else {
			cost = 5 * 50 + 5 * 10 + (hours - 10) * 30;
		}
	}
	printf("The total cost is %d DA\n", cost); 
	
	return 0;
}
		 
\end{lstlisting}
	\vspace{1cm}

\section*{Exercise 2}
\vspace{0.5cm}

\textbf{Code:} 
\begin{lstlisting}
#include<stdio.h>

int main() {
	int dep_hours,dep_mins,ari_hours,ari_mins,travel_hours,travel_mins;
	scanf("Enter the departure time and the arrival time : %d %d %d %d", &dep_hours, &dep_mins, &ari_hours, &ari_mins);
	if(dep_hours < 0 || dep_hours > 23 || dep_mins < 0 || dep_mins > 59 || ari_hours < 0 || ari_hours > 23 || ari_mins < 0 || ari_mins > 59) {
		printf("Error\n");
	} else {
		if(dep_hours <= ari_hours) {
			travel_hours = ari_hours - dep_hours;
			if(dep_mins <= ari_mins) {
				travel_mins = ari_mins - dep_mins;
			} else {
				travel_mins = 60 - dep_mins + ari_mins;
				travel_hours--;
			}
		} else {
			travel_hours = 24 - dep_hours + ari_hours;
			if(dep_mins <= ari_mins) {
				travel_mins = ari_mins - dep_mins;
			} else {
				travel_mins = 60 - dep_mins + ari_mins;
				travel_hours--;
			}
		}
	}
	printf("The total travel time is %d hours and %d minutes\n", travel_hours, travel_mins);
	
	return 0;
}
	
\end{lstlisting}
\vspace{1cm}

\section*{Exercise 3}
\vspace{0.5cm}

\textbf{Code:} 
\begin{lstlisting}
#include<stdio.h>

int main() {
	//Version 1
	int n;
	printf("Enter an integer : ");
	scanf("%d", &n);
	if(n % 2) {
		printf("The number %d is odd\n", n);
	} else {
		printf("The number %d is even\n", n);
	}
	//Version 2
	if (n & 1) {
		printf("The number %d is odd\n", n);
	} else {
		printf("The number %d is even\n", n);
	}
	return 0;
}
\end{lstlisting}
\vspace{1cm}

\section*{Exercise 4}
\vspace{0.5cm}

\textbf{Code:}
\begin{lstlisting}
#include<stdio.h>

int main() {
	float average;
	printf("Enter the average of the student : ");
	scanf("%f", &average);
	if(average < 0 || average > 20) {
		printf("Error");
	} else {
		if(average <= 9.99) {
			printf("Echec\n");
		} else if(average <= 13) {
			printf("Passable\n");
		} else if(average <= 15.99) {
			printf("Bien\n");
		} else if(average <= 19) {
			printf("Tres Bien\n");
		} else {
			printf("Excellent\n");
		}
	}
}
\end{lstlisting}
\vspace{1cm}
\section*{Exercise 5}
\vspace{0.5cm}

\textbf{Code:}
\begin{lstlisting}
#include<stdio.h>

int main() {
	int n;
	printf("Enter an integer between 100 and 999 : ");
	if(n < 100 || n > 999) {
		printf("Error \n");
	} else {
		int first_digit = n / 100;
		int last_digit = n % 10;
		if(first_digit == last_digit) {
			printf("The number is palindrome \n");
		} else {
			printf("The number is not palindrome \n");
		}
	}
	return 0;
}
\end{lstlisting}
\vspace{1cm}
\section*{Exercise 6}
\vspace{0.5cm}

\textbf{Code:} 
\begin{lstlisting}
#include<stdio.h>

int main() {
	int a,b,c,first,second,third;
	printf("Enter three integers : ");
	scanf("%d %d %d", &a, &b, &c);
	
	if(a >= b) {
		first = a;
		if(b >= c) {
			second = b;
			third = c;
		} else {
			second = c;
			third = b;
		}
	} else {   
		if(a >= c) {
			second = a;
			first = b;
			third = c;
		} else {
			second = c;
			first = b;
			third = a;
		}
	}
	printf("The integers in ascending order are : %d %d %d\n", third, second, first);
	
	return 0;
}
\end{lstlisting}
\vspace{1cm}
\newpage
\section*{Exercise 7}
\vspace{0.5cm}

\textbf{Code:} 
\begin{lstlisting}
#include<stdio.h>

int main() {
	float a,b,res;
	char op;
	printf("Enter an operation: ");
	scanf("%f %c %f", &a, &op, &b);
	
	if(op == '/' && b == 0) {
		printf("Division by zero is not allowed\n");
	} else {
		if(op == '+') { res = a+b; printf("Result: %f\n", res);}
		else if (op == '-') {res = a-b; printf("Result: %f\n", res);}
		else if (op == '*') {res = a*b; printf("Result: %f\n", res);}
		else if (op == '/') {res = a/b;  printf("Result: %f\n", res);}
		else printf("Invalid operation\n");
		
	}
	return 0;
}
\end{lstlisting}
\vspace{1cm}



\end{document}