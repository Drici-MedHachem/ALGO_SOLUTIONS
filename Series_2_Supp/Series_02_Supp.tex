\documentclass[12pt]{article}
\usepackage{graphicx} % For including images
\usepackage{fancyhdr} % For custom headers and footers
\usepackage{amsmath} % For math equations
\usepackage{listings} % For code listings
\usepackage{xcolor} % For coloring code
\usepackage{geometry} % For adjusting page margins
\geometry{a4paper, margin=1in}

% Define the header and footer
\pagestyle{fancy}
\fancyhf{}
\fancyhead[L]{\textbf{Drici Mohammed Hachem - Group A5}}
\fancyfoot[C]{\thepage}

% Define code listing style for C
\lstset{
	language=C, % Set language to C
	basicstyle=\small\ttfamily, % Basic font style
	numbers=left, % Add line numbers on the left
	numberstyle=\tiny\color{gray}, % Style for line numbers
	backgroundcolor=\color{white}, % Background color
	keywordstyle=\color{blue}, % Color for keywords (e.g., int, if, return)
	commentstyle=\color{green}, % Color for comments (e.g., // or /* ... */)
	stringstyle=\color{red}, % Color for strings (e.g., "Hello, World!")
	showstringspaces=false, % Don't show spaces in strings
	frame=single, % Add a frame around the code
	breaklines=true, % Automatically break long lines
	tabsize=4, % Set tab size to 4 spaces
	captionpos=b, % Caption position (bottom)
	xleftmargin=0pt, % Remove left margin
	framexleftmargin=0pt, % Align frame with the left margin
	morekeywords={printf, scanf, main, include} % Add additional C-specific keywords
}

\begin{document}
	
	% First Page
	\begin{titlepage}
		\centering
		\vspace*{2cm}
		
		\textbf{\Large People's Democratic Republic of Algeria}\\
		\vspace{0.5cm}
		\textbf{\large Echahid Hamma Lakhdar University}\\
		
		\vspace{1cm}
		\includegraphics[width=10cm]{university_logo.png} % Replace with your university logo file name
		
		\vspace{1.5cm}
		\rule{\textwidth}{1pt}
		\vspace{0.5cm}
		
		\textbf{\Large DS \& ALGO : Series 02}\\
		\vspace{0.5cm}
		\rule{\textwidth}{1pt}
		\vspace{0.5cm}
		
		\vspace{1.5cm}
		\textbf{\Huge Drici Mohammed Hachem}\\
		\vspace{0.5cm}
		\textbf{\Large Group A5}\\
		
		\vfill
	\end{titlepage}
	
	% Second Page
	\newpage
	\section*{Exercise 1}
	\vspace{0.5cm}
	
	\textbf{Code:}
\begin{lstlisting}
#include<stdio.h>

int main() {
    int i= 1;
    while(i <= 10) printf("%d \n",i++); 
    return 0;
}
\end{lstlisting}
	\vspace{1cm}
	
	\section*{Exercise 2}
	\vspace{0.5cm}
	
	\textbf{Code:}
\begin{lstlisting}
#include<stdio.h>

int main() {
    int i;
    for(i = 1 ; i <= 100; i++) printf("%d \n",i);
    return 0;
}
\end{lstlisting}
	\vspace{1cm}
	
	\section*{Exercise 3}
	\vspace{0.5cm}
	
	\textbf{Code:}
\begin{lstlisting}
#include<stdio.h> 

int main() {
    int i = 2;
    do {
        printf("%d \n",i);
        i += 2;
    }while(i <= 20);
    return 0;
}
\end{lstlisting}
	\vspace{1cm}
	
	\section*{Exercise 4}
	\vspace{0.5cm}
	
	\textbf{Code:}
\begin{lstlisting}
#include<stdio.h>

int main() {
    int n,i;
    do {
        printf("Enter a strictly positive integer n :");
        scanf("%d",&n);
    }while(n < 0);
    if (n % 2 == 1) i = n; else  i = n+1;
    printf("First %d \n",n+99);
    for(i ; i <= n+99 ; i += 2) printf("%d \n",i);
    return 0;

}
\end{lstlisting}
	\vspace{1cm}
	
	\section*{Exercise 5}
	\vspace{0.5cm}
	
	\textbf{Code:}
\begin{lstlisting}
#include<stdio.h>

int main() {
    int n,i = 2;
    do{
        printf("Enter a strictly positive integer n : ");
        scanf("%d",&n);
    }while(n < 0);
    
    while(i <= n) {
        printf("%d \n",i);
        i += 2;
    }
    return 0;
}
\end{lstlisting}
	\vspace{1cm}
\newpage
	\section*{Exercise 6}
	\vspace{0.5cm}
	
	\textbf{Code:}
\begin{lstlisting}
#include<stdio.h>

int main() {
    int n,i;
    do{
        printf("Enter a strictly positive integer n : ");
        scanf("%d",&n);
        if(n < 0) printf("Error! n must be strictly postiive");

    }while(n < 0);
    for( i = 1 ; i <= 9 ; i++) {
        printf("%d x %d  = %d \n", n , i , n*i);
    }
    return 0;
}
\end{lstlisting}
	\vspace{1cm}
	
	\section*{Exercise 7}
	\vspace{0.5cm}
	
	\textbf{Code:}
\begin{lstlisting}
#include<stdio.h>

int main() {
    int f_0 = 0;
    int f_1 = 1;
    int n,f_n ,temp,i =2;
    do {
        printf("Enter the which fibonnaci term to calculate : ");
        scanf("%d",&n);

    }while(n < 2);

    printf("F(0) = 0 \nF(1) = 1 \n");
    while(i< n) {
        f_n = f_0 + f_1;
        f_0 = f_1;
        f_1 = f_n;
        printf("F(%d) = %d \n",i,f_n);
        i++;
    }    
    return 0;
}
\end{lstlisting}
	\vspace{1cm}
	
	\section*{Exercise 8}
	\vspace{0.5cm}
	
	\textbf{Code:}
\begin{lstlisting}
#include<stdio.h>
int main() {
    int n , i;
    float sum = 0.0;
    do {
        printf("Enter tehe number of terms to calculate in the sum : ");
        scanf("%d",&n);
    }while (n < 1);
    for( i = 0 ; i <= n ; i++) {
        sum += 1/i;
    }
    printf("The harmonic sum up to %d  = %f",n,sum);
    return 0;
}
\end{lstlisting}
	\vspace{1cm}
	
	\section*{Exercise 9}
	\vspace{0.5cm}
	
	\textbf{Code:}
\begin{lstlisting}
#include<stdio.h>
int main() {
    int n , i;
    do {
        printf("Enter a natural number n : ");
        scanf("%d",&n);
    }while (n < 1);
    printf("The divisiors of %d are : 1 ",n);
    for(i = 2 ; i < n/2 ; i++ ) {
        if(n % i == 0) printf("%d ",i); 
    }
    printf("%d",n);
}
\end{lstlisting}
	\vspace{1cm}
\newpage
	\section*{Exercise 10}
	\vspace{0.5cm}
	
	\textbf{Code:}
\begin{lstlisting}
#include<stdio.h>

int main() {
    //Same as problem 8 ? 
}
\end{lstlisting}
	\vspace{1cm}
	
	\section*{Exercise 11}
	\vspace{0.5cm}
	
	\textbf{Code:}
\begin{lstlisting}
#include<stdio.h>

int main() {
    int n,i,sum = 0;
    do {
        printf("Enter the term to calculate the sum of the first n odd squares : ");
        scanf("%d",&n);
    }while(n < 1);
    for(i = 1 ; i <= n ; i++) {
        sum += (2*i - 1) * (2*i -1);
    }
    printf("The sum of the first %d odd squares = %d",n,sum);
    return 0;
}
\end{lstlisting}
	\vspace{1cm}
	
	\section*{Exercise 12}
	\vspace{0.5cm}
	
	\textbf{Code:}
\begin{lstlisting}
#include<stdio.h>

int main() {
    //Same as problem 12 ?
}
\end{lstlisting}
	\vspace{1cm}
	
	\section*{Exercise 13}
	\vspace{0.5cm}
	
	\textbf{Code:}
\begin{lstlisting}
#include<stdio.h>

int main() {
    int n,i, sum = 0;
    do {
        printf("Enter a natural number n : ");
        scanf("%d",&n);
    }while (n < 0);
    i = n;
    while(0 < i) {
        sum += i % 10;
        i /= 10;
    }
    printf("The sum of digits of %d = %d",n,sum);
    return 0;
}
\end{lstlisting}
	\vspace{1cm}
	
	\section*{Exercise 14}
	\vspace{0.5cm}
	
	\textbf{Code:}
\begin{lstlisting}
#include<stdio.h>

int main() {
    int cnt = 1,n,i = 3,j,isPrime = 1,sum;
    do {
        printf("Enter the number n to calculate the sum of primes up to n : ");
        scanf("%d",&n);
    } while(n < 1);
    if(n == 1) {
        sum = 2;
    }  else {
        sum = 2;
        while(cnt <= n) {
            for(j = 2 ; j < n/2 ; j++) {
                if(i % j == 0) {
                    isPrime = 0;
                }
            }
            if(isPrime) {
                sum += i;
                cnt++;
            }

            isPrime = 1;
            i++;
        }
    }
    print("The sum of the first %d prime numbers : %d",n,sum);
    return 0;

}
\end{lstlisting}
	\vspace{1cm}
	
	\section*{Exercise 15}
	\vspace{0.5cm}
	
	\textbf{Code:}
\begin{lstlisting}
#include<stdio.h>

int main() {
    int n,i,sum = 0;
    do {
        printf("Enter the term to calculate the sum of the first n odd cubes : ");
        scanf("%d",&n);
    }while(n < 1);
    for(i = 1 ; i <= n ; i++) {
        sum += (2*i - 1) * (2*i -1) * (2*i-1);
    }
    printf("The sum of the first %d odd cubes = %d",n,sum);
    return 0;
}
\end{lstlisting}
	\vspace{1cm}

\end{document}